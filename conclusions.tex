We describe a new method to model patient disease dynamics with several key features.  First, we apply continuous-time Markovian models for modeling disease dynamics, which avoids some of the limitations of discrete-time approaches when a dynamic process evolves at different time granulations, and when observations are irregularly sampled and missing not at random.  Second, non-parametric Bayesian clustering methods avoid the problem of identifying the number of clusters a priori, inferring the appropriate number of mixture component as a function of the sample size.

 The limitations of this work are mainly attributed to the temporal modeling steps. Continuous-time models bring us closer to a natural representation, but they are still inconsistent with the real-world. For example, the in the model the instantaneous probability for a state transition is the same for the entire duration of occupation.  Another issue is that not all patient models converged during the abstraction step.  Although this impacted only small fraction of the total patients, it is a key limitation to the method. 

  Immediate next steps are to extend temporal abstraction to continuous-time HMMs.  Also, we feel that external validation metrics for cluster assessment are fundamentally weak for many problems where clusters are not categorical, and rather a graded interval. In terms of intrinsic evaluation, heuristics such as the silhouettes are also limited.  Instead of developing one more metric, we propose a visualization tool to enable the browsing of temporal clustering results and feel this would be more useful for system development and is another direction for our future work.

