We apply our clustering approach to two fully de-identified clinical data sets.  Each consists of arbitrarily sampled clinical data in the low frequency setting, spanning both short and multi-year patient observation durations, and subject to missing data.  We describe the clustering application and the nature of the data in this section.

\subsection{Liver Decline in Hepatitis Patients}
Our first data set relates to liver disease.  A liver biopsy is the gold standard for the prognosis and treatment of liver fibrosis and important for the health provider and the patient to guide management and treatment of hepatitis B and C. In addition to the invasive nature of liver biopsy, which involves extracting a tissue sample of at least 23 cm in length, obtained with a 16-gauge needle inserted between two of the patients ribs, it is costly, and associated with complications that can be potentially life-threatening.  Also, it is subject to diagnostic error. For these reasons, alternatives for assessing the stage of liver fibrosis are in great demand and the lack of alternative assessment methods has been noted as a major limitation in both management and research in liver diseases.

The first data set consists of blood inspection and urinalysis laboratory data that was provided by the Chiba University Hospital in Japan, and was used for the ECML/PKDD-2003 and 2005 Discovery Challenges.  One objective of this shared task was to evaluate if laboratory examinations can be used to estimate the stage of liver fibrosis. The data set consists of recorded data for 771 patients of type B and C, spanning the years 1982 through 2001.  For many of the test types, values for low, normal and high values are indicated.

At the time of the challenge, medical research suggested some lab tests such as platelet count (PLT) were correlated at the time of biopsy; however, temporal analysis of the PLT data was rarely performed and limited by difficulties in time series comparison, irregular sampling intervals, and variable sequence lengths~\cite{Hirano05}.  Using PLT lab tests, one system~\cite{Hirano07} demonstrated that medically relevant time series features associated with the progression of liver fibrosis could be learned from the patient records.  Other lab tests reported by challenge participants as informative for predicting fibrosis stage included: ZTT, ALB, D-BIL and CHE.

\subsection{Inpatient Glucose Testing}
Our second data set relates to glucose tests.  It contains patients admitted to [omitted] Hospital with at least one physician ordered glucose test indicated in their EHR. Similar to the hepatitis patient data, the glucose time series presents methodological challenges in that it is irregularly sampled, and variable in length.  An additional complicating factor is an increased probability of record incompleteness.

National estimates report a 8.3\% prevalence of diabetes in the Unites States, with over seven million undiagnosed~\cite{cdc}.  The disease can result in various health complications such as kidney failure and blindness.  However, medical research shows that behavioral changes and other interventions can prevent or delay diabetes onset showing how importance of early diagnosis and treatment. Our glucose data set is derived from administrative data and our clustering application seeks to identify high-level patterns in physicians orders for glucose tests that corresponds with diabetes related symptoms.

Glucose tests are commonly ordered by physicians for hospitalized patients, and some hospitalises now suggest an initial test during admission should be part of standard care guidelines.  A single order with out any follow up on succeeding days may corresponds with an one-day admission, or with a longer stay that does not require ongoing glucose
monitoring. For this reason, a series of contiguous testing patterns for a patient suggests that their blood-glucose levels are being actively monitored by the attending physician.

For each patient that may or may not have diabetes, the sequence begins with the first physician ordered test on record and indicates the presence `1' or absence `0' of a physician's orders for each successive days, tracking multiple hospitalization and discharge periods.  A patent's series ends when a censoring state is encountered (i.e. death).


For our experiments we select patients with a time series length in the range of 1000 to 1025 days, and consist of 1024 patient 0/1 measurement sequences.  Our methods can be applied to larger subsets of the data set, but these constraints help visually assess results.  Other subsets we have experimented with but are not reported in this work are random collections or patients and those with that fall within a specific visit range.
