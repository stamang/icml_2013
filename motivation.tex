The most significant issues facing the US health care system in the coming years includes: major aging, the massive growth of chronic diseases, and not enough caregivers.  Increasing health care costs and quality issues already pose substantial issues and unless sustainable solutions can be developed our the US healthcare system will become increasingly stressed.

One proposition for transforming healthcare is to make it `data-driven' and has resulted in numerous government and private initiatives aiming to make meaningful use of digital health data.  Advocates hope that the richness of available patient data contained in these collections will enable a feedback loop of new knowledge discovery and translation to practice, supporting the engineering of a better and better health care system. There is an urgent need to demonstrate the cost-benefit of maintaining petabytes of patient data, and an important role for probabilistic learning algorithms that can assist in the discovery new knowledge from these noisy, heterogenous, fragmented data collections.

Here we provide one way to approach
data-driven care.  To model chronic disease dynamics, which may evolve slowly, over years, we extend the semiparametric clustering framework of Jebara et al.~\cite{JebSonTha07a} for learning patient and population level disease characteristics from arbitrarily sampled longitudinal patient data.  Specifically, we use parametric models based on continuous time Markov models paired with nonparmetric Bayesian clustering and we show results for two distinct data sets.

We describe the background for our work in Section~\ref{prelims}.  Our contribution and methods are detailed in Section~\ref{sp}.  The rest of the paper describes: two applications in Section~\ref{expts}, experimental results in Section~\ref{clin}, and conclusions for our work in Section~\ref{conc}.
