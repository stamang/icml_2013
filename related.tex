
%semi-parametric framework
Generative probabilistic frameworks for abstracting time series for standard multi-variate techniques have been broadly described~\cite{Cadez00ageneral}, and are motivated by the limitations of distance functions to approximate the variable amounts of temporal available for comparing observations in complex problems.  Semi-parametric clustering can be viewed as a specification in this broad category that uses DBNs, commonly hidden Markov models and their variants, to embed parametric assumptions about time and the relationships among variables to represent the underlying dynamical systems being studied.  Most applications pair abstractions with spectral clustering, and recent applications to motion capture data~\cite{JebSonTha07a} show performance comparable to that of supervised learning.

%timeseries using PGMs for abstraction
 Recent work applying probabilistic machine learning applications for the temporal modeling of high and medium frequency physiological signals collected in the critical care environment have demonstrated prognostic relevance.  Recent work used a nonparametric Bayesian framework to discover dynamical features from time series using a Bayesian non-parametric framework to capture higher-level concepts related to infant mortality, discovering dynamical features from time series~\cite{Saria09}. Also for unsupervised pattern discovery, other work used probabilistic clustering models to address uncertain and sparse sampled time series data~\cite{Marlin12}.  However, chronic disease progression can take weeks, months or years to manifest and these methods do not address the low frequency setting where trends instead of significant features can have increased importance, and observations are recorded over much longer durations.